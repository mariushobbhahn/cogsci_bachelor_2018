
\begingroup

\noindent
\textbf{\LARGE Abstract}\\
\parskip 12pt \\
Training a forward model and using temporal gradients has been successfully deployed in the domain of motion control. Inverse classification is a new technique in which a generative model is used both as a generator and a classifier. The generative model is first used to produce sequences given one-hot vectors representing classes as input. In the second step the network is fed a uniformly distributed vector that is iteratively adapted with the gradients resulting from the loss between the generated sequence and a target sequence. In the optimal case, the input converges towards the one-hot vector representing the class of the target sequence. To test the performance of inverse classification multiple experiments were conducted on sequences representing handwritten characters. \\
The technique of inverse classification is able to compete with a standard classification approach for less complex tasks but performs significantly worse on data of higher complexity.
\endgroup

\pagebreak

\begingroup

\noindent
\textbf{\LARGE Zusammenfassung}\\
\parskip 12pt \\
Das Training eines Vorwärtsmodells und die anschließende Nutzung von Gradienten wurde bereits erfolgreich im Bereich der Bewegungssteuerung eingesetzt. Inverse Klassifizierung ist eine neuartige Technik, die ein generatives Modell sowohl als Generator als auch zur Klassifizierung nutzt. Das Vorwärtsmodell wird erst darauf trainiert Sequenzen, die Klassen repräsentieren, zu generieren. Im Klassifizierungsschritt wird dem Netzwerk zunächst ein gleichverteilter Vektor gefüttert, welcher iterativ mit den Gradienten angepasst wird. Der Gradient entsteht aus dem Fehler zwischen dem Ergebnis des Vorwärtsprozesses und dem zu klassifizierenden Ziel. Im Optimalfall konvergiert der Eingangsvektor gegen den one-hot Vektor, der die Klasse der Zielsequenz repräsentiert. Um die Leistung der neuen Technik zu testen, wurden verschiedene Experimente auf einem Datensatz von Sequenzen, die handgeschriebenen Buchstaben entsprechen, ausgeführt.\\
Die Technik der Inversen Klassifizierung liefert bei weniger komplexen Aufgaben Resultate, die vergleichbar zu einer Standard Klassifizierung sind, aber schneidet bei höherer Komplexität merkbar schlechter ab. 
\vspace*{\fill}

\endgroup